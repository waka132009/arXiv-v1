% %% main.tex - AASTeX701 two-column final pass
\documentclass[twocolumn]{aastex701}

% ----- Packages (minimal) -----
\usepackage{amsmath, amssymb}
\usepackage{bm}
\usepackage{graphicx}
\graphicspath{{figs/}}
\usepackage[section]{placeins}
\usepackage{placeins}
\usepackage{needspace}
\usepackage{dblfloatfix}
\usepackage{balance}
\usepackage{flafter} % 図が参照より前に出ない保険

% 二段図の許容量を増やす(団子解消に効く)
\setcounter{dbltopnumber}{2}
\renewcommand{\topfraction}{.85}
\renewcommand{\floatpagefraction}{.7}

\makeatletter
\def\fps@figure{!htbp}
\def\fps@table{!htbp}
\makeatother

\hypersetup{colorlinks=true,linkcolor=blue,citecolor=blue,urlcolor=blue}
%\usepackage{array}
%\newcolumntype{L}[1]{>{\raggedright\arraybackslash}p{#1}}
% ----- Float tuning (tighter) -----
\setcounter{topnumber}{4}
\setcounter{bottomnumber}{3}
\setcounter{totalnumber}{6}
\setcounter{dbltopnumber}{1}

% 混雑ページだけ局所的にフロート数を絞る環境
\newenvironment{CrowdedFloats}{%
  \begingroup
  \setcounter{topnumber}{1}%
  \setcounter{totalnumber}{2}%
}{\endgroup}
% ===== end =====

\renewcommand{\topfraction}{0.97}
\renewcommand{\bottomfraction}{0.95}
\renewcommand{\textfraction}{0.05}
\renewcommand{\floatpagefraction}{0.70}
\renewcommand{\dbltopfraction}{0.88}
\renewcommand{\dblfloatpagefraction}{0.70}

\setlength{\floatsep}{4pt plus 1pt minus 1pt}
\setlength{\textfloatsep}{6pt plus 1pt minus 1pt}
\setlength{\intextsep}{4pt plus 1pt minus 1pt}
\setlength{\abovecaptionskip}{2pt}
\setlength{\belowcaptionskip}{1pt}

\raggedbottom

% ----- Macros -----
\newcommand{\ath}{a_{\rm th}}
\newcommand{\rg}{r_g}
\newcommand{\LEdd}{L_{\rm Edd}}
\newcommand{\Ltot}{L_{\rm tot}}
\newcommand{\FeKa}{\mathrm{Fe\,K}\alpha}
\newcommand{\epscoup}{\epsilon_{\rm coup}}
\newcommand{\aeq}{a_{\rm eq}}

% ---- Figure callout ----
\newcommand{\figcallout}[2]{%
  \noindent\begingroup\small\itshape
  \textbf{Figure~\ref{#1}} — #2\par
  \endgroup\vspace{2pt}%
}

% ----- PDF bookmark-safe strings -----
\PassOptionsToPackage{unicode}{hyperref}
\pdfstringdefDisableCommands{%
  \def\ath{a\_th}\def\rg{r\_g}\def\LEdd{L\_Edd}\def\Ltot{L\_tot}\def\FeKa{Fe K\string\alpha}%
  \def\epscoup{epsilon\_coup}\def\aeq{a\_eq}\def\mathrm#1{#1}%
}

% ----- Metadata -----
\shorttitle{Self-Sustained Penrose Excitation}
\shortauthors{Wakabayashi}

\begin{document}

\title{Self-Sustained Penrose Excitation of Accretion Disks:\\
A Spin-Regulated Mechanism for Super-Eddington Quasar Luminosities}
\correspondingauthor{Jun Wakabayashi}

\author[0009-0008-1891-4579]{Jun Wakabayashi}
\affiliation{Independent Researcher, Japan}
\email{waka132009@gmail.com}

\begin{abstract}
We present a high-spin, near-threshold equatorial radiative channel that (i) sustains super-Eddington luminosities and (ii) naturally enforces a sub-unity spin ceiling, explaining why spins do not exceed unity in practice. Compactness-limited transparency ($\ell$) gates the observable power, and a simple torque balance links the luminosity to the same control that sets the ceiling. The framework is immediately testable via three co-occurring hooks at high spin and high accretion rate: (1) a 20--120 keV high-energy shoulder favored over a featureless continuum by $\Delta$AIC or $\Delta$BIC $\ge 6$; (2) lag hardening from X to UV/optical; and (3) equator-aligned X-ray polarization with a rising degree. The path to falsification is explicit: in vetted high-spin, high-$\lambda$ quasars, failure to recover at least two hooks---after excluding heavy absorption, extreme inclination, and high-compactness cases---would refute the scenario.
\end{abstract}

\keywords{Quasars --- Accretion, accretion disks --- Black hole physics --- Relativistic processes}


\section{Introduction}\label{sec:intro}
\noindent\textbf{Threshold, transparency, and a natural spin ceiling.}
Near-threshold activation at high spin, gated by compactness-limited transparency ($\ell\!\lesssim\!30$), selects an equatorial radiative channel that both sustains $L/L_{\rm Edd}>1$ and imposes a spin ceiling.
This section sketches the mechanism and sets up testable consequences with explicit falsification (Eqs.~\eqref{eq:eps}--\eqref{eq:Ltot}; Fig.~\ref{fig:onset}).

Recent reviews have noted persistent tensions between thin–disk expectations and observations—
e.g., systematically large continuum-emitting sizes from microlensing and reverberation, and
wavelength-dependent lags exceeding simple reprocessing models \citep[e.g.,][]{Cackett2021RevMap,Sun2020AGNsize}.
These tensions motivate considering \emph{additional inner-disk heating} beyond local viscous dissipation.
Near $a_*\!\to\!1$, an \emph{equatorial, magnetically assisted Penrose-like split} provides a \emph{testable route}
to such reheating (Sec.~\ref{sec:mech-sketch}), with observational hooks and falsification criteria
(Sec.~\ref{sec:obs}).


\paragraph{Scope and modeling stance.}
We present a phenomenological framework: the microphysics of beam formation, transport, and equatorial deposition is intentionally kept agnostic, while the \emph{observable consequences} of such coupling are made explicit and falsifiable. This stance allows clean confrontation with data now and provides a clear interface to future GRMHD/plasma studies that could instantiate the coupling agent.

\paragraph{Boundary conditions are not fixed near $a_*\!\to\!1$.}
Much of the literature treats the inner disk, magnetosphere, and ergoregion with effectively fixed cross-component boundary conditions. We instead emphasize that near-extremal spin the boundaries themselves evolve: frame dragging enlarges and reshapes the ergoregion, the ISCO approaches it, and the equatorial return path becomes thin and resistive. In this coupled regime an equatorial current sheet is generically required by the global field topology; shear and flux loading drive plasmoid-dominated reconnection; and the resulting split redistributes $(E,L)$ so that a negative-energy branch is absorbed by the hole while a gain branch vents as a narrow equatorial quasi-beam, part of which returns to heat the inner disk.


\section{Theoretical Framework}\label{sec:theory}

\noindent All elements invoked below act outside the event horizon; the coupling operates in the ergoregion and deposits energy in the equatorial flow, avoiding assumptions about interior or singularity-scale physics.
\noindent\textit{Abbreviations and references.}
We use \emph{FFE} for the force-free, magnetically dominated limit ($\rho_e\mathbf{E}+\mathbf{J}\!\times\!\mathbf{B}\approx 0$, $\mathbf{J}\!\cdot\!\mathbf{E}=0$) and \emph{GRMHD} for ideal magnetohydrodynamics evolved on a Kerr background \citep{Komissarov2004MNRAS,GrallaJacobson2014MNRAS,EastYang2018PRD,Pan2018PRD}.
We use \emph{PIC} for first-principles kinetic simulations that resolve reconnection and plasmoid formation in the ergospheric current sheet \citep{Parfrey2019PRL,Bransgrove2021PRL}.
Throughout we group references as (FFE/GRMHD) versus (PIC).


\subsection{Kerr Energy Reservoir}\label{sec:kerr}


For a Kerr black hole of mass $M$ and spin $a_\ast$, the extractable rotational energy is
\Needspace{12\baselineskip}
\begin{equation}
E_{\rm rot}(a_\ast)=\Bigg[1-\sqrt{\tfrac{1}{2}\Big(1+\sqrt{1-a_\ast^2}\Big)}\Bigg]\,Mc^2,
\label{eq:Erot}
\end{equation}
reaching $\sim0.29\,Mc^2$ as $a_\ast\to1$.

\subsection{Spin-Triggered Coupling}\label{sec:coupling}
We posit a threshold spin $\ath$ above which an ergoregion-coupled agent activates. The effective coupling follows
\Needspace{7\baselineskip}
\begin{equation}
\epscoup(a_\ast)=
\begin{cases}
0, & a_\ast\le \ath,\\[3pt]
\epsilon_{\max}\Big(\dfrac{a_\ast-\ath}{1-\ath}\Big)^{n}, & a_\ast>\ath,
\end{cases}
\label{eq:eps}
\end{equation}
where $\epsilon_{\max}$ is the saturation cap and $n$ controls activation sharpness.
% === Figures tied to 3.2 (param space + onset) ===
\begin{figure}[!htbp]
  \centering
  \includegraphics[width=.92\linewidth]{Fig_Onset.pdf}
\caption{\textbf{Activation turns on steeply above the threshold.}
Peak $\Ltot/\LEdd$ rises rapidly once $a_\ast\!>\!\ath$, setting the lever for luminosity and spin regulation (cf.\ Eqs.~\eqref{eq:eps}--\eqref{eq:Ltot}). Dashed line: $\,\ath=0.97$.}
  \label{fig:onset}
\end{figure}
% --- Takeaway first (3行以内) ---
\noindent\textit{Activation threshold.}
The equatorial coupling turns on steeply once the spin exceeds a threshold \(a_{\rm th}\).
In our fiducial calibration, the rise is rapid near \(a_\ast \simeq 0.97\), setting the lever for both luminosity boost and spin regulation (cf.\ Eqs.~\eqref{eq:eps}--\eqref{eq:Ltot}).
\FloatBarrier
Unless noted, we adopt the following fiducials for figures and estimates: $M=10^9M_\odot$, $\eta_{\rm acc}=0.1$, $\epsilon_{\max}=0.1$, $n=2$, $R_0=10^2$, and $\alpha=50$; the illustrative threshold is $\ath\simeq0.97$.
\noindent\emph{Micro-bridge.}
While we keep microphysics agnostic, the fiducial range
$\epsilon_{\rm coup}\sim10^{-2}$–$10^{-1}$ is broadly consistent with
energy-release fractions seen in near-horizon, plasmoid-dominated
reconnection in recent GRMHD studies. For the present framework we only
assume that $\epsilon_{\rm coup}$ rises sharply above $a_{\rm th}$ and
saturates below $\epsilon_{\max}$; detailed calibration is left for
future simulations.

\paragraph{Phenomenological parameters and physical ranges.}
The cap $\epsilon_{\max}$ limits equatorial deposition efficiency by energy-budget and pair-compactness constraints; $n$ controls activation sharpness above $\ath$ as an effective criticality index of the coupling geometry. $(R_0,\alpha)$ regulate the dissipation footprint to maintain transparency, trading compactness against reprocessing. We restrict these to physically plausible ranges and view them as interfaces for future GRMHD/plasma calibration, not curve-fitting knobs.
\paragraph{Sensitivity and saturation.}
The activation $\epsilon_{\rm coup}(a_*)$ is intentionally steep: $(a_{\rm th},n)$ set the narrowness of the “spin ceiling”. In practice $\epsilon_{\rm coup}$ and the leakage fraction depend nonlinearly on flux loading and $\dot M$, with possible saturation arbitrarily close to the extremal limit. We therefore treat $(a_{\rm th},n,\epsilon_{\max},\alpha)$ as calibration parameters to be fixed by future GRMHD-in-ergoregion testbeds.


\paragraph{Power partition.}
We allow the extracted rotational power to partition between a polar BZ jet and an equatorial feedback channel:
\[
P_{\rm ext}=P_{\rm jet}+P_{\rm eq},\qquad f_{\rm BZ}+f_{\rm eq}=1.
\]
Near $a_*\!\to\!1$, $f_{\rm eq}$ can become substantial (radio-quiet, radiation-dominated states), while $f_{\rm BZ}$ dominates in radio-loud systems; hybrid states are possible in transient MAD-like regimes.
%\FloatBarrier
\subsection{Mechanism sketch: a leaky equatorial return path}\label{sec:mech-sketch}
\paragraph{Projectile sufficiency.}
Reconnection ejecta (plasmoids) in the ergoregion carry specific energy $E$ and angular momentum $L$. A Penrose split requires a branch with $E-\Omega_H L<0$ (horizon condition).\footnote{Notation follows Fig.~\ref{fig:equatorial_trigger}.}
Tension-driven redistribution during ejection changes $L$ by $\Delta L\!\sim\!\mathcal{O}(r_{\rm g}v_\phi)$ and reconnection outflows can reach $v\!\sim\!0.1$–$0.5c$ in high-$S$ sheets; hence tens-of-percent shifts in $L$ suffice to place a fraction on the negative-energy branch, with the complement forming the \emph{positive-energy branch} (equatorial quasi-beam).
Near-extremal spin brings the ISCO into close contact with the ergoregion and enforces an equatorial return current sheet at their interface. Shear and flux loading thin the sheet until it becomes tearing-unstable; reconnection then ejects plasmoids as a chain of narrow, equator-following pulses. These ejecta provide the “projectile” required by a Penrose-like energy split: within the ergoregion, a fraction of the flow is placed on negative-energy-at-horizon trajectories while the counterpart gains energy and escapes. The extraction condition is simply
\begin{equation}
E - \Omega_H L \;<\; 0, \label{eq:penrose-cond}
\end{equation}
so the black hole’s rotational energy pays for the escaping branch. The centrifugal barrier and toroidal tension form an equatorial nozzle, so a growing fraction of the circuit power vents as a collimated equatorial quasi-beam while the DC return still closes globally. A modest back-flow coupling ($\epsilon_{\rm coup}\!\sim\!10^{-2}$–$10^{-1}$) suffices to heat the inner disk and regenerate magnetic flux, closing a self-sustained loop. The active zone and leakage fraction increase monotonically with spin (and also depend nonlinearly on flux loading and accretion rate), with possible saturation arbitrarily close to the extremal limit.
% 節頭直後に図が先出しされるのを抑止したい場合は次を追加:
% \suppressfloats[t]
\begin{figure*}[p!]
  \centering
  \includegraphics[width=\textwidth]{fig_equatorial_trigger_schematic.pdf}
  \caption{\textbf{Equatorial Penrose trigger: negative vs.\ positive branches.}}
  Plasma from the inner disk/plunging region enters the ergosphere and splits kinematically:
  a \emph{negative-energy branch} with $E-\Omega_H L<0$ falls through the \emph{event horizon} ($r_{\rm H}$),
  while a \emph{positive-energy branch} emerges as an equatorial quasi-beam that reheats the inner rim near $r_{\rm ISCO}$.
  Landmarks—ergosurface, $r_{\rm ISCO}$, and the equatorial return current sheet (blue ribbon)—are indicated; branches are shown as \emph{red solid} (positive) and \emph{blue dashed} (negative) to match Eq.~\eqref{eq:penrose-cond}.
  \par\noindent\footnotesize (FFE/GRMHD: \citealt{Komissarov2004MNRAS,EastYang2018PRD,Pan2018PRD}; PIC: \citealt{Parfrey2019PRL,Bransgrove2021PRL}.)
  \label{fig:equatorial_trigger}
\end{figure*}
% 小節を跨いで流れないように必要なら1回だけ:
\FloatBarrier

\paragraph{Why an equatorial, quasi-collimated branch?}
Near $a_*\!\to\!1$ the combination of (i) the equatorial centrifugal barrier,
(ii) strong toroidal fields generated by shear (hoop stress), and
(iii) pressure deficits carved by intermittent reconnection outflows
forms an ``equatorial nozzle''. The \emph{positive-energy branch} therefore propagates as a
narrow, radiation-supported quasi-beam rather than escaping along the poles.
Because the transport remains largely collisionless/Poynting-dominated until
it reprocesses in the inner disk, and because the dissipation footprint is
extended ($R_{\rm eff}\!\sim\!10^{2}$–$10^{3}\,r_g$), the pair compactness
along the beam stays low, consistent with the transparency requirement of Eq.~(6).

%\FloatBarrier
\subsection{Self-Sustained Penrose Excitation}\label{sec:penrose}
The mean extraction power is $\langle P_{\rm ext}\rangle=E_{\rm rot}/\tau$, and with duty cycle $d$ the instantaneous power during active phases is $P_{\rm ext}\sim\langle P_{\rm ext}\rangle/d$. Here $\tau$ denotes the effective extraction e-folding timescale of the rotational-energy reservoir, i.e., $\langle P_{\rm ext}\rangle=E_{\rm rot}/\tau$ using Eq.~\eqref{eq:Erot}.
%\FloatBarrier
\subsection{Disk Dissipation and Scale}\label{sec:diss}
A fraction $\epscoup(a_\ast)$ of $P_{\rm ext}$ is deposited into the disk:
\begin{equation}
L_{\rm self}=\epscoup\,P_{\rm ext},
\label{eq:Lself}
\end{equation}
and the dissipation spreads over an effective radius
\paragraph{Self-transparency (physical note).}
The scaling $R_{\rm eff}\!\approx\!R_0[1+\alpha(\epsilon_{\rm coup}/\epsilon_{\max})]\,r_{\rm g}$ phenomenologically captures geometric spreading, multi-zone deposition, and increased scattering mean free paths as the equatorial quasi-beam heats and rarefies the inner disk corona. Our results require $R_{\rm eff}\!\sim\!10^2$–$10^3\,r_{\rm g}$ during bright episodes to keep the pair compactness low (cf. Sec.~\ref{sec:transp}).
\begin{equation}
R_{\rm eff}\approx R_0\Big[1+\alpha\Big(\frac{\epscoup}{\epsilon_{\max}}\Big)\Big]\,\rg,
\label{eq:Reff}
\end{equation}
with $\rg=GM/c^2$, fiducial $R_0\sim10^2$ and $\alpha\sim50$.

%\FloatBarrier
\subsection{Transparency (Compactness Constraint)}\label{sec:transp}
Transparency requires pair compactness
\begin{equation}
\ell=\frac{L_{\rm self}\sigma_T}{4\pi R_{\rm eff}\,m_ec^3}\lesssim30,
\label{eq:ell}
\end{equation}
which couples Eqs.~\eqref{eq:Lself} and \eqref{eq:Reff} and motivates $R_{\rm eff}\sim10^2$--$10^3\,\rg$ during bright episodes.  
\textit{We adopt $\ell\!\lesssim\!30$ as a conservative transparency threshold following classic compactness arguments; the precise value depends on geometry and spectrum and can be re-tuned in data applications. See, e.g., \citep{1984MNRAS.209..175S,LightmanZdziarski1987}.}

%\FloatBarrier
\subsection{Spin and Mass Evolution}\label{sec:evol}
The horizon angular frequency is
\begin{equation}
\Omega_H=\frac{a_\ast c^3}{2GM\big(1+\sqrt{1-a_\ast^2}\big)},
\label{eq:OmegaH}
\end{equation}
where $r_H=\rg\big(1+\sqrt{1-a_\ast^2}\big)$ and $\rg=GM/c^2$.
Evolution obeys
\begin{align}
\frac{dM}{dt}&=\frac{dM_{\rm acc}}{dt}-\frac{P_{\rm ext}}{c^2},\label{eq:dM}\\
\frac{dJ}{dt}&=\frac{dJ_{\rm acc}}{dt}-\frac{P_{\rm ext}}{\Omega_H},\label{eq:dJ}\\
\frac{da_\ast}{dt}&=\frac{c}{GM^2}\frac{dJ}{dt}-2a_\ast\frac{1}{M}\frac{dM}{dt}.\label{eq:da}
\end{align}

% === Figures tied to 3.6 (torque balance + spin track) ===
\begin{CrowdedFloats}
\begin{figure}[!htbp]
  \vspace*{2pt}
  \centering
  \includegraphics[width=.95\linewidth]{Fig3a.pdf}\\[4pt]
  \includegraphics[width=.95\linewidth]{Fig3b.pdf}
  \caption{\textbf{A spin ceiling emerges from torque balance.}
  (a) $\dot a$ crosses zero where Eqs.~\eqref{eq:dM}--\eqref{eq:da} with $\Omega_H$ from Eq.~\eqref{eq:OmegaH} balance the torques. (b) The implied equilibrium spin $\aeq$ increases with coupling $\epscoup$; the shaded band marks a plausible range.}  \label{fig:fig3}
\end{figure}

\begin{figure}[!htbp]
  \centering
  \includegraphics[width=\columnwidth,height=0.36\textheight,keepaspectratio]{fig2a_spin_evolution.pdf}
  \caption{\textbf{Near-threshold systems hover close to the ceiling.}
  Time evolution of $a_\ast$ from Eqs.~\eqref{eq:dM}--\eqref{eq:da} shows self-regulated drift toward and around $\aeq$.}
  \label{fig:spin-evo}
\end{figure}
\end{CrowdedFloats}
%\FloatBarrier
\subsection{Net Luminosity}\label{sec:lum}
\begin{equation}
\Ltot=L_{\rm acc}+L_{\rm self},\qquad
L_{\rm acc}\approx \eta_{\rm acc}\,\Big(\frac{dM_{\rm acc}}{dt}\Big)c^2,
\label{eq:Ltot}
\end{equation}
which links the dynamical solution (Eqs.~\eqref{eq:dM}--\eqref{eq:da}) to observables.

\paragraph{Energetic sanity check (one-line).}
With $M_{\rm BH}=10^9\,M_\odot$ [$Mc^2\simeq1.8\times10^{63}$ erg] and $E_{\rm rot}\!\sim\!0.1\,Mc^2$, a reservoir e-fold $\tau=10^7$ yr, $\epscoup=0.05$, and duty $d=0.2$ yield 
$L_{\rm self}\!\sim\!(E_{\rm rot}/\tau)\,(\epscoup/d)\!\approx\!1.4\times10^{47}\,{\rm erg\,s^{-1}}\!\sim\!1.1\,\LEdd$,
and $2$–$3\,\LEdd$ when combined with concurrent accretion, while $da_\ast/dt<0$ prevents overspin.

% === Figures tied to 3.7 (timescale + L mapping) ===
%\Needspace{18\baselineskip}$
\begin{figure}[!htbp]
  \vspace*{2pt}
  \centering
  \includegraphics[width=.95\linewidth]{Fig4a.pdf}\\[4pt]
  \includegraphics[width=.95\linewidth]{Fig4b.pdf}
  \caption{\textbf{Timescales and luminosities follow the coupling law.}
  (a) Equilibration time $\tau_{\rm eq}$ versus coupling $\epsilon_{\rm coup}$.
  (b) $L/L_{\rm Edd}$ versus equilibrium spin $a_{\rm eq}$, with points labeled by $\epsilon_{\rm coup}$ (cf.\ Eqs.~(1), (2), (11)).}
  \label{fig:fig4}
\end{figure}

\section{Results}\label{sec:results}
The figures embedded in Sec.~\ref{sec:theory} visualize each theoretical ingredient immediately after the defining equations to avoid misalignment between formulae and diagnostics. Here we summarize cross-implications and observational hooks.

\clearpage
\subsection{Observational Hooks}\label{sec:obs}
\suppressfloats[t]  % avoid a float jumping above the heading

\noindent\textit{Why this matters.} What to look for and how to falsify, in plain terms.

% --- Quick recall of the equations each plot is tied to (往復を減らす) ---
\noindent\textit{Recall.}
Activation $\epsilon_{\rm coup}(a_\ast)$ and power $P_{\rm eq}=\epsilon_{\rm coup}\,\dot{M}c^2\,g(\ell)$
(cf.\ Eqs.~\eqref{eq:eps}, \eqref{eq:Ltot}), with transparency set by compactness $\ell$ (Eq.~\eqref{eq:ell}).
These establish the trends in the plots below.

We highlight three testable signatures expected to co-occur at high spin and high Eddington ratio under transparency ($\ell\!\lesssim\!30$):
(i) a hard X-ray high-energy excess (“shoulder”) favored over a featureless continuum by $\Delta$AICc or $\Delta$BIC $\ge 6$ in the observed 20–120 keV band (rest $\sim$0.3–1 MeV at $z\gtrsim6$);
(ii) lag hardening—X$\to$UV/optical lags increase with photon energy;
(iii) equator-aligned X-ray polarization with rising degree at high spin/high $\lambda$.
\textbf{Falsify} if $\ge2$ of these are absent in a high-spin, high-$\lambda$ sample after excluding heavy absorption/extreme inclination and $\ell\!\gtrsim\!30$ cases.
% --- 図はトップ優先で近くに出す ---
\begin{figure}[t!]
  \vspace*{2pt}
  \centering
  \includegraphics[width=.49\columnwidth]{Fig5a.pdf}
  \hfill
  \includegraphics[width=.49\columnwidth]{Fig5b.pdf}
  \caption{\textbf{What each hook should look like (linked to the equations).} 
  (a) Fe K$\alpha$ lag (in $r_g/c$) versus $a_{\rm eq}$. 
  (b) Observable fraction (hard X/MeV or polarization) versus $a_{\rm eq}$.}
  \label{fig:fig5}
\end{figure}

%\FloatBarrier
\noindent\textit{At a glance.} How power repartitions at high $\lambda$ across mechanisms.

\floattable
\begin{deluxetable*}{lccc}[t] % ← ★付き+[t]で上部固定(専用ページなら [p])
\tablenum{1}
\tablecaption{\textbf{Bottom line:} the equatorial channel sustains $L/L_{\rm Edd}\!>\!1$ \emph{and} imposes a spin ceiling $a_{\rm eq}\!\approx\!a_{\rm th}$, complementary to slim-disk (no ceiling) and BZ/MAD (polar power, $a_{\rm eq}\!\sim\!0.5$–$0.7$).\label{tab:mech-contrast}}
\tabletypesize{\footnotesize}
\tablewidth{0pt}
\tablehead{
  \colhead{} & \colhead{$L/L_{\rm Edd}>1$} & \colhead{spin ceiling} & \colhead{channel}
}
\startdata
Slim disk        & yes (advective)                  & no                               & radiation (disk) \\
BZ/MAD           & indirect                         & yes ($a_{\rm eq}\sim0.5$--$0.7$) & polar Poynting (jet) \\
\textbf{This work} & \textbf{yes (equatorial)}         & \textbf{yes ($a_{\rm eq}\approx a_{\rm th}$)} & \textbf{equatorial Penrose-like} \\
\enddata
\end{deluxetable*}
% (必要なら直前に一度だけ) 
\FloatBarrier
%\suppressfloats[b]

\clearpage
%\Needspace{10\baselineskip} 
\section{Discussion}\label{sec:discussion}
These bounds align with Fig.~\ref{fig:param-a}--\ref{fig:param-b} and are sufficient to reach $L/\LEdd\sim2$--$3$ with rarer $5$--$10\times$ excursions. Slim disks allow modest super-Eddington flows but no spin ceiling. BZ/MAD explain jet power yet not the radiative dominance of radio-quiet quasars. This framework ties hyperluminous output and the spin ceiling via near-extremal, equatorial coupling.
Consistent with GRFFE/GRMHD studies, an \emph{equatorial return-current sheet forms within the ergoregion} and becomes plasmoid-unstable; detached plasmoids naturally split into branches with $E-\Omega_H L\lessgtr 0$ (captured vs.\ escaping), enabling rotational-energy extraction via a Penrose-like radiative channel \citep{Komissarov2004MNRAS,EastYang2018PRD,Pan2018PRD,Parfrey2019PRL,Bransgrove2021PRL}.


%\medskip
\noindent\textit{Relation to Blandford--Znajek jets.}
BZ is polar/Poynting-dominated; our mechanism is equatorial/radiative. Hybrid states and an anti-correlated see-saw are expected; counterexamples (simultaneously strong jets and high radiative output) can occur in transitional MAD-like regimes.

% --- Parameter-space (panel a) : make it slightly shorter to allow text insertion ---
\begingroup
\setlength{\textfloatsep}{6pt plus 2pt minus 2pt} % 図まわりの余白を局所的に詰める
\begin{figure}[tp!]
  \centering
  \includegraphics[width=0.95\columnwidth,trim=0 6 0 4,clip]{fig10a_Reff100.pdf}
  \vspace{-2pt}
  \caption{\textbf{Transparency selects the viable high-luminosity regime (a).}
  For $R_{\rm eff}=100\,r_g$, the allowed region lies \emph{outside} $\ell{=}30$ (black), where peak $L_{\rm tot}/L_{\rm Edd}$ (color) reaches $2$--$3$ with rarer $5$--$10\times$ excursions.}
  \label{fig:param-a}
\end{figure}
\endgroup
% --- Parameter-space (panel b) : make it slightly shorter to allow text insertion ---
\begin{figure}[tp!]
  \centering
  \includegraphics[width=0.95\columnwidth,trim=0 6 0 4,clip]{fig10b_Reff1000.pdf}
  \vspace{-2pt}
  \caption{\textbf{A larger dissipation footprint widens the transparent window (b).}
  For $R_{\rm eff}=1000\,r_g$, the $\ell{=}30$ boundary (black) shifts so that higher peaks in $L_{\rm tot}/L_{\rm Edd}$ remain compactness-safe.}
  \label{fig:param-b}
\end{figure}

\FloatBarrier


\paragraph{Power partition between polar and equatorial channels.}
We model the spin-extraction power as a partition between a polar Blandford--Znajek (BZ) channel and an equatorial feedback channel,
% amsmath は AASTeX で既定読み込み。無ければ \usepackage{amsmath}
\begin{align}
P_{\rm tot} &= P_{\rm BZ} + P_{\rm eq}, \\
P_{\rm BZ}  &= f_{\rm BZ}\,P_{\rm ext}, \qquad
P_{\rm eq} = f_{\rm eq}\,P_{\rm ext}, \\
f_{\rm BZ} + f_{\rm eq} &= 1.
\end{align}

This partition predicts an anti-correlation between radio-jet dominance and equatorial reheating diagnostics within the high-$a_*$, high-$\lambda$ subset, with hybrid states possible in transient MAD phases.
Our claim is modest: in the high-spin, high-Eddington subset, $f_{\rm eq}$ is statistically non-zero and sometimes dominant; in others, $f_{\rm BZ}$ may prevail. Outside this subset we make no claim.

\subsection*{Validation roadmap: observations and simulations}
\noindent\textbf{Observational fronts.}
(i) \textit{Polarization vs.\ spin}: search for a step-like rise of optical/UV polarization fractions and azimuthal rotations across $a_\ast\simeq \ath$. (ii) \textit{Lag broadening}: Fe\,K$\alpha$ and BLR reverberation lags should broaden in high $L/\LEdd$ episodes at near-threshold spins. (iii) \textit{Spectral compactness}: hard X/MeV excesses consistent with low pair-compactness at $R_{\rm eff}\!\sim\!10^{2}$–$10^{3}\,r_g$. (iv) \textit{See-saw with BZ}: anti-correlation between radio-jet dominance and equatorial reheating diagnostics, with transitional hybrid states.

\noindent\textbf{Simulation fronts.}
(i) GRMHD-in-ergoregion testbeds to calibrate the effective coupling law $\epscoup(a_\ast)$ and its slope $n$; (ii) radiative transfer with pair kinetics to refine the $\ell$ threshold; (iii) semi-analytic population modeling to predict the duty distribution $d$ vs.\ spin. These can be plugged into the present framework without altering its falsifiable structure.

\paragraph{What would falsify this framework (kill shots).}
We explicitly delineate outcome patterns under which the equatorial channel is unnecessary. In the high-spin, high-$\lambda$ subset (e.g., $a_*\gtrsim0.8$, $\lambda\!\equiv\!L_{\rm bol}/L_{\rm Edd}\gtrsim0.3$), any of the following, if established \emph{as a population trend}, would falsify our claim:
\begin{enumerate}
\item Systematic absence of the EUV/soft-X/MeV excess (no ``shoulder'') \emph{and} systematically weak high-ionization lines (He\,II, N\,V, C\,IV) relative to the parent population.
\item Reverberation lags from X$\rightarrow$UV showing no energy dependence (no inward reheating signature).
\item $L_{\rm bol}$ not anti-correlated (even weakly) with $a_*$ within the bright subset (no self-regulated spin-down imprint).
\item Optical/X-ray polarization failing to favor equatorial angles or degrees when near-threshold spins are inferred.
\end{enumerate}
If \emph{two or more} items above hold simultaneously for the same high-spin, high-$\lambda$ subset, we would consider $f_{\rm eq}\!\to\!0$ supported and this mechanism unnecessary.

\section{Conclusion}\label{sec:conclusion}
Self-sustained Penrose excitation near extremal Kerr SMBHs can inject rotational energy into the accretion disk, yielding sustained $2$--$3\times$ Eddington with rarer $5$--$10\times$ episodes while enforcing a spin ceiling.

% ---------------- Appendix ----------------
\clearpage
\appendix

\floattable
\begin{deluxetable*}{lcccc}[p]   % [t]で上部固定。収まらなければ[p]に
\tablewidth{0pt}
\tabletypesize{\normalsize}           % 大きく見せたい:\normalsize / 収めたい:\scriptsize
\tablecaption{\textbf{Candidate Gallery} — Representative objects; values are indicative.\label{tab:cands}}
\tablehead{
  \colhead{Name} & \colhead{$z$} &
  \colhead{$M_{\rm BH}/M_\odot$} & \colhead{$L/L_{\rm Edd}$} &
  \colhead{Notes}
}
\startdata
TON 618        & 2.219 & $\sim6.6\times10^{10}$ & $\sim3$     & \parbox[t]{0.36\textwidth}{Extremely massive; radio-loud; literature refs.\tablenotemark{a}}\\
J2157-3602     & 4.75  & $\sim3.4\times10^{9}$  & $\gtrsim10$ & \parbox[t]{0.36\textwidth}{Hyper-luminous; super-Eddington episode indications.\tablenotemark{b}}\\
J0100+2802     & 6.30  & $\sim1.2\times10^{10}$ & $\sim2$     & \parbox[t]{0.36\textwidth}{$z>6$ luminous quasar.\tablenotemark{c}}\\
J0439+1634     & 6.51  & $\sim7\times10^{9}$    & $\sim2{-}3$ & \parbox[t]{0.36\textwidth}{Possible lensing history.\tablenotemark{d}}\\
\enddata
\tablecomments{Values are illustrative; per-object sourcing is out of scope.}
\tablenotetext{a}{Virial-factor dominated systematics; radio-loud bias possible.}
\tablenotetext{b}{Bolometric/line-width systematics $\gtrsim 0.3$ dex.}
\tablenotetext{c}{High-$z$ mass methods differ (reflection vs.\ continuum).}
\tablenotetext{d}{Historical lensing debate; values assume de-lensing consensus.}
\end{deluxetable*}

\FloatBarrier
\section*{Cautionary note.}
Claims here are modular and falsifiable. Alternative mechanisms may dominate in other classes.

\section*{Author contributions}
Conceptualization, modeling, analysis, visualization, and writing: J.~Wakabayashi.

\section*{Competing interests}
The author declares no competing interests.

\section*{Data and code availability}
All figures can be regenerated from scripts in the accompanying repository; data sources and acquisition steps are documented in a README.\\
An archived OSF snapshot (Version 1.1) is available: \href{https://doi.org/10.17605/osf.io/62gzv}{doi:10.17605/osf.io/62gzv}.
\section*{Communication and media}
Media note: The paper and its reproducibility package are the sole authoritative sources.

\begin{acknowledgments}
This work stands on decades of insight into black–hole accretion, spin, and energy extraction. I am indebted to the community that built the modern framework of quasar physics—from classical Penrose energy extraction and disk theory to spin–jet coupling and polarimetry—and to teams who made public data and tools available.
\\[2pt]
\textit{Use of large language models.} Large language model assistants (Google Gemini and OpenAI ChatGPT) were used for drafting support (editing for clarity, formatting suggestions, and figure placement/LaTeX troubleshooting). No novel data, equations, or results were produced by these tools. No confidential or unpublished data were provided to them. All analysis, derivations, and conclusions are by the author, who takes full responsibility for the content; the models are not authors.
\\[2pt]
I also thank colleagues and readers who provided critical comments on early drafts. Any remaining errors are mine.
\end{acknowledgments}

\software{\texttt{latexmk}, \texttt{AASTeX701}}

\nocite{*}
%\balance
\bibliographystyle{aasjournalv7}
\bibliography{main}

\end{document}


\section{Observational Hooks and Falsification}
\suppressfloats[t]  % ← このページ上部でのフロート先出しを抑止
\noindent\textit{Why this matters.} What to look for and how to falsify, in plain terms.
We highlight three testable signatures expected to co-occur at high spin and high Eddington ratio under transparency ($\ell\!\lesssim\!30$):
\begin{itemize}
\item \textbf{Hard-X shoulder}: a high-energy excess favored over a featureless continuum by $\Delta$AICc or $\Delta$BIC $\ge 6$ in 20--120 keV (obs).
\item \textbf{Lag hardening}: X$\to$UV/optical lags increase with photon energy.
\item \textbf{Equator-aligned polarization}: PA parallel to the disk and degree rising with spin/$\lambda$.
\end{itemize}
\textbf{Falsify} if $\ge2$ of the above are absent in a high-spin, high-$\lambda$ sample after excluding heavy absorption/extreme inclination and $\ell\!\gtrsim\!30$ cases.

\section{Position in Context (Concise)}
We briefly situate the proposed equatorial Penrose-like coupling relative to BZ/MAD jets, reconnection-driven coronal heating, and reverberation/polarimetry constraints; an expanded survey is provided in the OSF Appendix.

