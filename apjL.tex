% ------------------------------
% ApJL V1 — Minimal Letter (≤3500 words, ≤5 figures/tables)
% ------------------------------
\documentclass[twocolumn]{aastex701}

% ----- Packages (minimal) -----
\usepackage{amsmath, amssymb}
\usepackage{bm}
\usepackage{graphicx}
\graphicspath{{figs/}}
\usepackage[section]{placeins}
\usepackage{placeins}
\usepackage{needspace}
\usepackage{dblfloatfix}
\usepackage{balance}
\usepackage{flafter} % 図が参照より前に出ない保険
\usepackage{orcidlink}


% 二段図の許容量を増やす(団子解消に効く)
\setcounter{dbltopnumber}{2}
\renewcommand{\topfraction}{.85}
\renewcommand{\floatpagefraction}{.7}

\makeatletter
\def\fps@figure{!htbp}
\def\fps@table{!htbp}
\makeatother

\hypersetup{colorlinks=true,linkcolor=blue,citecolor=blue,urlcolor=blue}
%\usepackage{array}
%\newcolumntype{L}[1]{>{\raggedright\arraybackslash}p{#1}}
% ----- Float tuning (tighter) -----
\setcounter{topnumber}{4}
\setcounter{bottomnumber}{3}
\setcounter{totalnumber}{6}
\setcounter{dbltopnumber}{1}

% 混雑ページだけ局所的にフロート数を絞る環境
\newenvironment{CrowdedFloats}{%
  \begingroup
  \setcounter{topnumber}{1}%
  \setcounter{totalnumber}{2}%
}{\endgroup}
% ===== end =====

\renewcommand{\topfraction}{0.97}
\renewcommand{\bottomfraction}{0.95}
\renewcommand{\textfraction}{0.05}
\renewcommand{\floatpagefraction}{0.70}
\renewcommand{\dbltopfraction}{0.88}
\renewcommand{\dblfloatpagefraction}{0.70}

\setlength{\floatsep}{4pt plus 1pt minus 1pt}
\setlength{\textfloatsep}{6pt plus 1pt minus 1pt}
\setlength{\intextsep}{4pt plus 1pt minus 1pt}
\setlength{\abovecaptionskip}{2pt}
\setlength{\belowcaptionskip}{1pt}

\raggedbottom

% ----- Macros -----
\newcommand{\ath}{a_{\rm th}}
\newcommand{\rg}{r_g}
\newcommand{\LEdd}{L_{\rm Edd}}
\newcommand{\Ltot}{L_{\rm tot}}
\newcommand{\FeKa}{\mathrm{Fe\,K}\alpha}
\newcommand{\epscoup}{\epsilon_{\rm coup}}
\newcommand{\aeq}{a_{\rm eq}}

% --- Compactness threshold (single source of truth) ---
\newcommand{\ellcrit}{30}
\newcommand{\ellhard}{\ell<\ellcrit}         % hard screen
\newcommand{\ellsoft}{\ell\lesssim\ellcrit}  % soft/typical statement

% tight enumerate/itemize (環境ラッパ・パッケージ不要)
\newenvironment{tenumerate}{%
  \begin{enumerate}\setlength{\itemsep}{0pt}\setlength{\parsep}{0pt}\setlength{\parskip}{0pt}%
}{\end{enumerate}}
\newenvironment{titemize}{%
  \begin{itemize}\setlength{\itemsep}{0pt}\setlength{\parsep}{0pt}\setlength{\parskip}{0pt}%
}{\end{itemize}}

% === one-liners(1つだけに統一) ===
% 置換(コロン版)
\DeclareRobustCommand{\tldr}[1]{%
  \noindent\textit{In brief:}~#1%
  \par\smallskip
  \noindent\rule{\columnwidth}{0.2pt}\par\medskip
}


\DeclareRobustCommand{\nav}[1]{%
  \noindent\textit{See also:}~#1\par\smallskip
}

% === Figure callout ===
\newcommand{\figcallout}[2]{%
  \noindent\begingroup\small\itshape
  \textbf{Figure~\ref{#1}}:\ #2\par
  \endgroup\vspace{2pt}%
}
% ----- PDF bookmark-safe strings -----
\PassOptionsToPackage{unicode}{hyperref}
\pdfstringdefDisableCommands{%
  \def\ath{a\_th}\def\rg{r\_g}\def\LEdd{L\_Edd}\def\Ltot{L\_tot}\def\FeKa{Fe K\string\alpha}%
  \def\epscoup{epsilon\_coup}\def\aeq{a\_eq}\def\mathrm#1{#1}%
}

% ----- Metadata -----
\shorttitle{Self-Sustained Penrose Excitation}
\shortauthors{Wakabayashi}

\begin{document}

% ---- 著者ブロックの最終修正案 (1:57 AMの提案) ----
\title{Self-Sustained Penrose Excitation of Accretion Disks:\\
A Spin-Regulated Mechanism for Super-Eddington Quasar Luminosities}
\correspondingauthor{Jun Wakabayashi}

\author{Jun Wakabayashi}
\affiliation{Independent Researcher, Japan}
\email{waka132009@gmail.com}
\orcidlink{0009-0008-1891-4579}





% ---- Abstract (≤ 250 words) ----
\begin{abstract}
Quasars often radiate at several to ten times the Eddington limit of their central SMBHs.
We propose a self-sustained, equatorial Penrose-like excitation that operates at the
ISCO–ergoregion interface just outside the horizon as the spin approaches unity ($a_*\!\to\!1$).
Even modest effective coupling ($\epsilon_{\rm coup}\!\sim\!10^{-2}$–$10^{-1}$) yields on average
$2$–$3\times\LEdd$, with occasional $5$–$10\times$ episodes when the dissipation footprint is extended
and transparent ($R_{\rm eff}\!\sim\!10^{2}$–$10^{3}\,r_g$). We adopt a phenomenological framework and
do not fix the microphysics; the results depend on a small set of effective parameters (plotted
luminosities are \textit{isotropic-equivalent} unless noted). Crucially, the mechanism is confined to
the ergoregion and equatorial disk—i.e., horizon-exterior and therefore empirically testable.
We further outline three observational hooks (a hard X-ray/MeV shoulder,
X-to-UV/optical lag hardening, and equator-aligned X-ray polarization)
and an explicit null criterion for high-spin, high-Eddington quasars.
\end{abstract}
\keywords{accretion, accretion disks — black hole physics — quasars: general}



% ---- 1. Introduction (4–6 paragraphs max) ----
\section{Introduction}\label{sec:intro}

\begin{figure}[htbp]
  \centering
  \includegraphics[width=0.9\linewidth]{sspe_infographic_p1_conundrum.pdf}
\caption{\textbf{The Quasar Conundrum.}
  Left: quasars exist with $L/L_{\rm Edd}>1$; a super-Eddington tail with $L/L_{\rm Edd}\gtrsim 2$ is implied.
  Right: observed spins cluster below the Thorne limit $a_\ast\simeq0.9985$ despite sustained accretion (spin ceiling).}
  \label{fig:conundrum}
\end{figure}

Observations of super-Eddington quasars ($L/L_{\rm Edd}>1$) and sub-unity spin ceilings ($a_*<1$) present a challenge to standard accretion models. Persistent tensions, such as large continuum-emitting sizes and complex wavelength-dependent lags, motivate the need for additional inner-disk heating mechanisms beyond local viscous dissipation (Fig.~\ref{fig:conundrum}).
We propose that these phenomena are consequences of a spin-regulated equatorial channel that activates as $a_\ast\to 1$.
Near-extremal spin brings the ISCO into close contact with the ergoregion, forcing an equatorial current sheet where plasmoid-dominated reconnection can occur.
We posit that these ejecta provide the necessary ``projectiles'' for a
magnetically assisted, Penrose-like energy split (cf. \citep{Penrose1969,Penrose2002}).
This process extracts the black hole's rotational energy and deposits it
back into the disk, sustaining $L/L_{\rm Edd}>1$ while the associated
counter-torque simultaneously imposes a natural sub-unity spin ceiling.
This Letter outlines the minimal framework for this mechanism, its spin-regulation dynamics (Sec.~\ref{sec:theory}), and a set of explicit, falsifiable observational hooks derived from it (Sec.~\ref{sec:hooks}).


% ---- 2. Minimal Framework ----
\section{Minimal Framework}\label{sec:theory}
We posit a phenomenological framework where a fraction of the black hole's rotational energy is extracted and deposited into the inner disk via an equatorial channel. This coupling activates steeply only near extremal spin, following a threshold law (Fig.~\ref{fig:onset})
\begin{equation}
  \epscoup(a_\ast)=
  \begin{cases}
    0, & a_\ast\le \ath,\\[3pt]
    \epsilon_{\max}\Big(\dfrac{a_\ast-\ath}{1-\ath}\Big)^{n}, & a_\ast>\ath,
  \end{cases}
  \label{eq:eps}
\end{equation}
% --- Figure 2 ---
\begin{figure}[!htbp]
  \centering
  \includegraphics[width=.92\linewidth]{sspe_infographic_p2_flowchart.pdf}
  \caption{\textbf{Self-Sustained Penrose Excitation (SSPE): mechanism sketch.}
  Rotational energy is tapped at the ISCO--ergoregion interface and thermally reinjected into the inner disk via equatorially collimated quasi-beams, sustaining $L/L_{\rm Edd}>1$ while regulating $a_\ast$.}
  \label{fig:flowchart}
\end{figure}

where $\ath$ is the threshold spin (fiducial $\ath\simeq0.97$), $\epsilon_{\max}$ is the saturation cap (fiducial $\sim0.1$), and $n$ controls the sharpness (fiducial $n=2$).
This process has two simultaneous consequences: (i) luminosity enhancement and (ii) spin regulation.

\noindent\textbf{(i) Luminosity.} The total luminosity becomes the sum of standard viscous accretion ($L_{\rm acc}$) and the self-sustained power ($L_{\rm self}$) from this mechanism:
\begin{equation}
\Ltot = L_{\rm acc} + L_{\rm self}(a_*),
\label{eq:Ltot_simple}
\end{equation}
where $L_{\rm self} = \epscoup(a_*) P_{\rm ext}$ and $P_{\rm ext}$ is the power extracted from the spin reservoir. This naturally sustains $L/L_{\rm Edd}>1$ when $a_* > \ath$.

\noindent\textbf{(ii) Spin Regulation.} The energy extraction applies a counter-torque to the hole (cf. $dJ/dt \propto -P_{\rm ext}/\Omega_H$) , forcing the spin to drift downward toward a sub-unity equilibrium value ($a_{\rm eq} < 1$).
 This provides a self-limiting ceiling.

\noindent\textbf{Observability Gate.} For the extracted power $L_{\rm self}$ to escape, the dissipation region must remain optically thin to pair production.
This requires the pair compactness $\ell$ to remain low (e.g., \citealt{1984MNRAS.209..175S,LightmanZdziarski1987}):
\begin{equation}
  \ell=\frac{L_{\rm self}\sigma_T}{4\pi R_{\rm eff}\,m_e c^3}
    \lesssim 30,
  \label{eq:ell}
\end{equation}
which implies an extended dissipation footprint ($R_{\rm eff}\sim10^2$--$10^3\,\rg$).
This compactness limit acts as a physical gate on the observable luminosity.

% --- Figure (Adopted from main.tex) ---
\begin{figure}[!htbp]
  \centering
  \includegraphics[width=.92\linewidth]{Fig_Onset.pdf}
  \caption{\textbf{Activation turns on steeply above the threshold $\ath$.}
  Peak $\Ltot/\LEdd$ (from Eq.~\ref{eq:eps}) rises rapidly once $a_\ast\!>\!\ath$ (fiducial $\ath=0.97$), setting the lever for both luminosity and spin regulation.}
  \label{fig:onset}
\end{figure}

% --- Figure (Hooks) ---
\begin{figure*}[htbp]
  \centering
  \includegraphics[width=0.8\textwidth]{sspe_infographic_p3_hooks.pdf}
  \caption{\textbf{The Three Observational Hooks.}
  From left to right: (1) A hard X-ray/MeV shoulder (20--120 keV) favored over a standard continuum. (2) Lag hardening (X$\to$UV/optical), where lag increases monotonically with energy. (3) X-ray polarization aligned with the equator, with a rising degree.}
  \label{fig:hooks}
\end{figure*}

% ---- 3. Observational Hooks & Null Criteria ----
\section{Observational Hooks and Null Criteria}\label{sec:hooks}

Three testable signatures (Fig.~\ref{fig:hooks}) are expected to co-occur in quasars
with near-extremal spin ($a_*\!\to\!1$) and high accretion rate ($L/L_{\rm Edd}>1$),
provided that the dissipation region remains pair-transparent ($\ell\lesssim\ellcrit$).
Taken together, these signatures are intended to provide compelling but
not definitive evidence for centrally triggered, equatorial inner-disk heating.


\noindent\textbf{Screens (apply first):}
To define a clean sample before searching for SSPE signatures, exclude
(i) heavily absorbed systems (Compton-thick and similar),
(ii) extreme inclinations where geometry is poorly constrained, and
(iii) high-compactness cases with $\ell\gtrsim\ellcrit$ where pair opacity would suppress
the emergent shoulder.

\noindent\textbf{Hooks (apply jointly under screens):}
\begin{titemize}
  \item \textbf{Hard X-ray/MeV shoulder (20--120~keV, obs):}
  model comparison favors a spectrum with an additional hard shoulder over a smooth
  cutoff continuum by $\Delta$AICc \emph{or} $\Delta$BIC $\ge 6$
  (rest-frame $\sim$0.3--1~MeV at $z\gtrsim6$).
  Such a statistically significant shoulder is difficult to obtain from a standard
  corona alone and points to an additional, centrally triggered hard component.

  \item \textbf{Lag hardening (X$\to$UV/optical):}
  energy-resolved lags increase \emph{monotonically} with photon energy within an epoch,
  i.e., $d\,{\rm lag}/d\log E>0$.
  This pattern is naturally produced when variability is initiated in a hotter inner
  region and propagates through reprocessing and scattering into lower-energy bands.

  \item \textbf{Equator-aligned X-ray polarization (2--10~keV):}
  the electric vector position angle is aligned with the equatorial plane
  (EVPA~$\parallel$~equator within $\sim\!15^\circ$), and the polarization degree
  increases toward higher spin and Eddington ratio.
  Such a configuration is more naturally explained by strong equatorial illumination
  and scattering than by a polar, jet-dominated geometry.
\end{titemize}

\noindent\textbf{Null criteria (falsification):} The SSPE framework
is constructed to be explicitly falsifiable.
In vetted high-spin, high-Eddington-ratio quasar samples
that pass the screens above, a systematic failure to detect
at least two of the three hooks (hard X-ray/MeV shoulder,
X-to-UV/optical lag hardening, and equator-aligned X-ray
polarization) would refute this scenario.


% ---- 4. Separation from Polar Power ----
\section{Separation from Polar Power}\label{sec:separation}
The proposed equatorial, radiative mechanism (SSPE) is distinct from the polar, Poynting-dominated Blandford--Znajek (BZ) channel.
We model the total extracted spin power as a partition
$P_{\rm ext}=P_{\rm BZ}+P_{\rm eq}$, where the fractions satisfy
$f_{\rm BZ}+f_{\rm eq}=1$. This framework naturally predicts an
anti-correlated ``see-saw'' between polar (jet) power and equatorial
(radiative) signatures within the high-spin, high-Eddington-ratio subset.
When $f_{\rm BZ}$ is large, jet power dominates and equatorialradiative hooks are comparatively weak.
Conversely, large $f_{\rm eq}$ corresponds to radio-quiet, radiatively dominated states that require
$L/L_{\rm Edd}>1$ and should preferentially exhibit the SSPE hooks outlined in Sec.~\ref{sec:hooks}.
% ---- 5. Conclusions ----
\section{Conclusions}\label{sec:conclusions}
We propose that self-sustained Penrose excitation (SSPE) near extremal Kerr SMBHs provides a viable channel to inject rotational energy into the inner accretion disk.
This mechanism naturally sustains super-Eddington luminosities (e.g., $L/L_{\rm Edd}\sim2$--$3$ or higher) while simultaneously enforcing a sub-unity spin ceiling via the associated counter-torque.
The framework is explicitly falsifiable: in vetted quasars with near-extremal spins and high Eddington ratios that satisfy the screening criteria above, a systematic absence of at least two of the three proposed observational hooks (a 20--120~keV shoulder, X-to-UV/optical lag hardening, and equator-aligned polarization) would refute this scenario.

\section*{Competing interests}
The author declares no competing interests.

\begin{acknowledgments}
This work stands on decades of insight into black-hole accretion, spin, and energy extraction. I am indebted to the community that built the modern framework of quasar physics\textemdash{}from classical Penrose energy extraction and disk theory to spin-jet coupling and polarimetry\textemdash{}and to teams who made public data and tools available.
Interactive visual summaries of the SSPE mechanism and the three observational hooks are available at
\url{https://waka132009.github.io/self_penrose_demo/interactive_paper.html}
and \url{https://waka132009.github.io/SSPE_infographic/infographic.html}.
\\[2pt]
\textit{Use of large language models.} Large language model assistants (Google Gemini and OpenAI ChatGPT) were used for drafting support (editing for clarity, formatting suggestions).  All analysis, derivations, and conclusions are by the author, who takes full responsibility for the content.
\\[2pt]
\end{acknowledgments}



\balance{}

\bibliographystyle{aasjournalv7}
\bibliography{apjL}

\nocite{*}

\end{document}

