## ApJL(V1).tex スケルトン(AASTeX701, two-column)

```tex
% ------------------------------
% ApJL V1 — Minimal Letter (≤3500 words, ≤5 figures/tables)
% ------------------------------
\documentclass[twocolumn]{aastex701}

% Packages (minimal)
\usepackage{amsmath, amssymb}
\usepackage{bm}
\usepackage{graphicx}
\graphicspath{{figs/}}
\usepackage[section]{placeins}
\usepackage{dblfloatfix}
\usepackage{flafter}
\hypersetup{colorlinks=true,linkcolor=blue,citecolor=blue,urlcolor=blue}

% Float tuning
\makeatletter
  \def\fps@figure{!htbp}
  \def\fps@table{!htbp}
\makeatother
\setcounter{topnumber}{4}
\setcounter{bottomnumber}{3}
\setcounter{totalnumber}{6}
\setcounter{dbltopnumber}{1}
\renewcommand{\topfraction}{0.97}
\renewcommand{\bottomfraction}{0.95}
\renewcommand{\textfraction}{0.05}
\renewcommand{\floatpagefraction}{0.70}
\renewcommand{\dbltopfraction}{0.88}
\renewcommand{\dblfloatpagefraction}{0.70}

% Macros (examples)
\newcommand{\LEdd}{L_{\rm Edd}}
\newcommand{\ath}{a_{\rm th}}
\newcommand{\epsc}{\epsilon_{\rm coup}}
\newcommand{\rg}{r_{\rm g}}

\shorttitle{Equatorial Penrose-like Coupling}
\shortauthors{Wakabayashi}

\begin{document}

\title{Self-Sustained Penrose Excitation of Accretion Disks:\\ A Spin-Regulated Mechanism for Super-Eddington Quasar Luminosities}

\author[0000-0008-1891-4579]{Jun Wakabayashi}
\affiliation{Independent Researcher, Japan}

% ---- Abstract (≤ 250 words) ----
\begin{abstract}
Quasars often radiate at several to ten times the Eddington limit of their central SMBHs.
We propose a self-sustained, equatorial Penrose-like excitation that operates at the
ISCO–ergoregion interface just outside the horizon as the spin approaches unity ($a_*\!\to\!1$).
Even modest effective coupling ($\epsilon_{\rm coup}\!\sim\!10^{-2}$–$10^{-1}$) yields on average
$2$–$3\times\LEdd$, with occasional $5$–$10\times$ episodes when the dissipation footprint is extended
and transparent ($R_{\rm eff}\!\sim\!10^{2}$–$10^{3}\,r_g$). We adopt a phenomenological framework and
do not fix the microphysics; the results depend on a small set of effective parameters (plotted
luminosities are \textit{isotropic-equivalent} unless noted). Crucially, the mechanism is confined to
the ergoregion and equatorial disk—i.e., it is horizon-exterior and therefore empirically testable;
we provide observational hooks and explicit falsification criteria.
\end{abstract}
% 1) Problem & gap
% 2) New channel (equatorial Penrose-like, ISCO–ergoregion interface)
% 3) Threshold law eps_coup(a*) and transparency guard l \lesssim 30
% 4) Minimal luminosity model L/LEdd = lambda0 + G * eps_coup(a*)
% 5) Falsifiable hooks (lags / polarization / MeV shoulder) & null criteria
% Keep \le 250 words.
\end{abstract}

\keywords{accretion, accretion disks — black hole physics — quasars: general}

% ---- 1. Introduction (4–6 paragraphs max) ----
\section{Introduction}
% Ultra-luminous quasars, spin ceiling, limits of viscous accretion (thin/slim)
% Motivate equatorial, self-sustained Penrose-like coupling

% ---- 2. Minimal Framework ----
\section{Minimal Framework}
We posit a near-equatorial coupling at the ISCO–ergoregion interface. The activation of the coupling efficiency follows a threshold law
\begin{equation}
\epsc(a_*) = \eps_{\max} \left( \frac{a_* - \ath}{1 - \ath} \right)^{n},\quad a_* > \ath,
\end{equation}
leading to a minimal luminosity model
\begin{equation}
\frac{L}{\LEdd} = \lambda_0 + G\,\epsc(a_*).\label{eq:Lmodel}
\end{equation}
A compactness guard limits observability,
\begin{equation}
\ell \simeq \frac{1836}{R_{\rm eff}/\rg}\,\frac{L}{\LEdd} \lesssim 30.
\end{equation}

% ---- 3. Observational Hooks & Null Criteria ----
\section{Observational Hooks and Null Criteria}
% Bulleted, concise
% - Energy-dependent Fe K\alpha / UV lags broadening
% - Polarization angle/degree vs spin (equatorial dominance)
% - Soft MeV shoulder
% Null: (i) persistent absence of the shoulder; (ii) no energy dependence in X->UV lags; (iii) no anti-correlation between L_bol and a*; (iv) isotropic/polar-dominated polarization in high-spin, high-\lambda samples

% ---- 4. Separation from Polar BZ Power ----
\section{Separation from Polar Power}
External power splits as $P_{\rm ext}=P_{\rm BZ}+P_{\rm eq}$; the equatorial fraction rises as $a_*\to1$ while respecting the transparency bound.

% ---- 5. Conclusions ----
\section{Conclusions}
% One-paragraph summary + immediate tests + link to OSF/GitHub for details

% ---- Figures (≤5 total) ----
% Fig1: Schematic (equatorial quasi-beam & power split)
% Fig2: Threshold activation eps_coup(a*) with sensitivity to a_th, n
% Fig3: L/LEdd vs a* with \ell=30 cap line (several R_eff)
% Fig4: Observational hooks panel (conceptual)

\acknowledgments
% Acknowledge informal feedback, tools, etc.

\software{AASTeX701, Python 3.12, Matplotlib}

\dataavailability{Supplementary materials at OSF DOI and project GitHub repository.}

\end{document}
```
